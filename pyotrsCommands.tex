% custom commands I use verry frequently. use % custom commands I use verry frequently. use % custom commands I use verry frequently. use % custom commands I use verry frequently. use \input{path/to/file/pyotrsCommands} to enable tham
% requirements: 
% not all of them might be required in the commands. i use them frequently so i just placed them here.
%    \usepackage{amsmath}
%    \usepackage{amssymb}
%    \usepackage{amsfonts}
%    \usepackage{amsthm}
%    \usepackage{cancel}
%    \usepackage{mathtools}
%    \usepackage{enumerate}
%
%
% Commands:
  % Sets and definitions
      %\set{Universum}{Prädikat} Intensionale Definition einer Menge
      \newcommand{\set}[2]{\left\{#1\middle|#2\right\}}
      
      %\listset{Elemente} Extensionale Definition einer Menge
      \newcommand{\listset}[1]{\left\{ #1 \right\}}

  % Intervals and Ranges
      \newcommand{\rn}[1]{\mathbb{R}^{#1}}
      \newcommand{\NN}{\mathbb{N}}
      \newcommand{\RR}{\mathbb{R}}
      \newcommand{\QQ}{\mathbb{Q}}
      \newcommand{\setminuszero}{\setminus\listset{0}}

    % special chars (my laptop keyboard doesn#t have these for some reason
      \newcommand{\lt}{<}
      \newcommand{\gt}{>}
      \newcommand{\absbar}{|}

    %\abs{Formel} Betragsklammern
    \newcommand{\abs}[1]{\left\absbar #1 \right\absbar}

    %\flg{name}{index} Folge
    \newcommand{\flg}[2]{\left(#1_{#2}\right)}
    \newcommand{\flgdef}[2]{\left(#1_{#2}\right)_{#2\in\NN}}

    %\inv Reziprogwert
    \newcommand{\inv}{^{-1}}

    %\max{Indexmenge}
    \renewcommand{\max}[1]{\underset{#1}{\mathrm{max}}}
 to enable tham
% requirements: 
% not all of them might be required in the commands. i use them frequently so i just placed them here.
%    \usepackage{amsmath}
%    \usepackage{amssymb}
%    \usepackage{amsfonts}
%    \usepackage{amsthm}
%    \usepackage{cancel}
%    \usepackage{mathtools}
%    \usepackage{enumerate}
%
%
% Commands:
  % Sets and definitions
      %\set{Universum}{Prädikat} Intensionale Definition einer Menge
      \newcommand{\set}[2]{\left\{#1\middle|#2\right\}}
      
      %\listset{Elemente} Extensionale Definition einer Menge
      \newcommand{\listset}[1]{\left\{ #1 \right\}}

  % Intervals and Ranges
      \newcommand{\rn}[1]{\mathbb{R}^{#1}}
      \newcommand{\NN}{\mathbb{N}}
      \newcommand{\RR}{\mathbb{R}}
      \newcommand{\QQ}{\mathbb{Q}}
      \newcommand{\setminuszero}{\setminus\listset{0}}

    % special chars (my laptop keyboard doesn#t have these for some reason
      \newcommand{\lt}{<}
      \newcommand{\gt}{>}
      \newcommand{\absbar}{|}

    %\abs{Formel} Betragsklammern
    \newcommand{\abs}[1]{\left\absbar #1 \right\absbar}

    %\flg{name}{index} Folge
    \newcommand{\flg}[2]{\left(#1_{#2}\right)}
    \newcommand{\flgdef}[2]{\left(#1_{#2}\right)_{#2\in\NN}}

    %\inv Reziprogwert
    \newcommand{\inv}{^{-1}}

    %\max{Indexmenge}
    \renewcommand{\max}[1]{\underset{#1}{\mathrm{max}}}
 to enable tham
% requirements: 
% not all of them might be required in the commands. i use them frequently so i just placed them here.
%    \usepackage{amsmath}
%    \usepackage{amssymb}
%    \usepackage{amsfonts}
%    \usepackage{amsthm}
%    \usepackage{cancel}
%    \usepackage{mathtools}
%    \usepackage{enumerate}
%
%
% Commands:
  % Sets and definitions
      %\set{Universum}{Prädikat} Intensionale Definition einer Menge
      \newcommand{\set}[2]{\left\{#1\middle|#2\right\}}
      
      %\listset{Elemente} Extensionale Definition einer Menge
      \newcommand{\listset}[1]{\left\{ #1 \right\}}

  % Intervals and Ranges
      \newcommand{\rn}[1]{\mathbb{R}^{#1}}
      \newcommand{\NN}{\mathbb{N}}
      \newcommand{\RR}{\mathbb{R}}
      \newcommand{\QQ}{\mathbb{Q}}
      \newcommand{\setminuszero}{\setminus\listset{0}}

    % special chars (my laptop keyboard doesn#t have these for some reason
      \newcommand{\lt}{<}
      \newcommand{\gt}{>}
      \newcommand{\absbar}{|}

    %\abs{Formel} Betragsklammern
    \newcommand{\abs}[1]{\left\absbar #1 \right\absbar}

    %\flg{name}{index} Folge
    \newcommand{\flg}[2]{\left(#1_{#2}\right)}
    \newcommand{\flgdef}[2]{\left(#1_{#2}\right)_{#2\in\NN}}

    %\inv Reziprogwert
    \newcommand{\inv}{^{-1}}

    %\max{Indexmenge}
    \renewcommand{\max}[1]{\underset{#1}{\mathrm{max}}}
 to enable tham
% requirements: 
% not all of them might be required in the commands. i use them frequently so i just placed them here.
%    \usepackage{amsmath}
%    \usepackage{amssymb}
%    \usepackage{amsfonts}
%    \usepackage{amsthm}
%    \usepackage{cancel}
%    \usepackage{mathtools}
%    \usepackage{enumerate}
%
%
% Commands:
  % Sets and definitions
      %\set{Universum}{Prädikat} Intensionale Definition einer Menge
      \newcommand{\set}[2]{\left\{#1\middle|#2\right\}}
      
      %\listset{Elemente} Extensionale Definition einer Menge
      \newcommand{\listset}[1]{\left\{ #1 \right\}}

  % Intervals and Ranges
      \newcommand{\rn}[1]{\mathbb{R}^{#1}}
      \newcommand{\NN}{\mathbb{N}}
      \newcommand{\RR}{\mathbb{R}}
      \newcommand{\QQ}{\mathbb{Q}}
      \newcommand{\setminuszero}{\setminus\listset{0}}

    % special chars (my laptop keyboard doesn#t have these for some reason
      \newcommand{\lt}{<}
      \newcommand{\gt}{>}
      \newcommand{\absbar}{|}

    %\abs{Formel} Betragsklammern
    \newcommand{\abs}[1]{\left\absbar #1 \right\absbar}

    %\flg{name}{index} Folge
    \newcommand{\flg}[2]{\left(#1_{#2}\right)}
    \newcommand{\flgdef}[2]{\left(#1_{#2}\right)_{#2\in\NN}}

    %\inv Reziprogwert
    \newcommand{\inv}{^{-1}}

    %\max{Indexmenge}
    \renewcommand{\max}[1]{\underset{#1}{\mathrm{max}}}
